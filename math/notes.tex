\documentclass{article}
\usepackage[utf8]{inputenc}
\usepackage[english]{babel}

\title{Math Advanced Topics Notes}

\newcommand{\Z}{$\mathbb{Z}$}

\usepackage{amsmath}
\usepackage{amssymb}
\usepackage{amsthm}



\theoremstyle{definition}
\newtheorem{definition}{Definition}[section]

\newtheorem{theorem}{Theorem}



\theoremstyle{remark}
\newtheorem*{remark}{Remark}

\author{Emmanuel Eppinger}

\date{\today}

\begin{document}
\maketitle
\section{Subrings}
\subsection{Subring Theorem}
A non-empty subset of a ring is a subring under the same operations if and only if it is closed under multiplication and subtraction.
\begin{proof}

Forward Direction: It is obviously true in the forwards direction.
Backwards Direction:
Suppose R is a ring and S a non-empty subset, which is closed under multiplication and subtraction. Now we wish to show that S is a ring. Now, because S is non-empty, we choose an arbitrary element in S: s. First note that because S is closed under subtraction, $s-s=0 \in S$. Next suppose $a \in S$, let because 0 is an element of S and S is closed under subtraction $0-a \in S$. Now suppose that $a,b \in S$. It has been previously proven that $-b \in S$. But then, $a+b=a-(-b) \in S$ which is shows that S is closed under addition.\\

To show that S is a r ring, need to show addition is commutative, addition and multiplication are associative, and multiplication is distributive over addition. But all of these properties are are based operations in R which is a ring. So we know it works.

\end{proof}

\subsection{Multiplicative Identity}
We call an element \textit{u} of a ring R a unity or multiplicative identity if $ua = au = a$ $\forall a \in R$. Note: This is not necessary to be a ring.

\subsection{Commutative Ring}
A ring where multiplication is commutative.

\section{Integral Domains and Fields}
\begin{definition}

Let R be a commutative ring. An element a, not zero, is a zero divisor if there exists another element b such that ab=0. Of course, b is also a zero divisor. \\

Examples: 2,3 for $\mathbb{Z}_6$\\
none for $\mathbb{Z}_5$\\
For $\mathbb{Z}x\mathbb{Z}$: $(a,0)$ and $(0,b)$ where a,b are integers and not zero
non for integers

\end{definition}

This gives us a definition:

\begin{definition}

A commutative ring with unity that has no zero divisor is called an \textit{Integral Domain} or simply a domain.

Ex.: \Z and \Z sub 5 are domains as are $\mathbb{Q}, \mathbb{R}, \mathbb{C}$

\end{definition}

Note: 2\Z is NOT a domain because it has no unity, even though it is commutative, but has no zero divisors.

\begin{theorem}
	Multiplicative Cancellation: Suppose R is an integral domain and a,b,c are elements of R, with a not zero. If ab=ac, then b = c.\\
    \begin{proof}
		Suppose that b is not equal to c. Since a is not equal to 0 and there are no zero divisors then ab is not equal to ac.
	\end{proof}
\end{theorem}

\subsection{Units}
\begin{definition}
Suppose R is a ring with unity 1. Let a be any on zero element of R. We say a is a unit if there is  an element b of R st $ab = ba = 1$. In this case, b is a multiplicative inverse of a. (of course, b is also a unit with inverse a.)
\end{definition}

Note: unity 1 is always a unit because $1 \cdot 1 = 1$.\\

Quick Exercise:\\
Determine the units of:
\begin{itemize}
\item In \Z
1,-1
\item In $\mathbb{Q}$
All non zero elements
\item In $\mathbb{R}$
All non zero elements
\item In $\mathbb{Z}_6$
1 and 5.
\end{itemize}

Quick Exercise:\\
Compute units of these rings:
\begin{itemize}
\item $\mathbb{Z}_5$: 1, 2, 3, 4
\item $\mathbb{Z}_{12}$: 1, 5, 7, 11
\item $\mathbb{Z}x\mathbb{Z}$: $(1,1)$, $(1,-1)$, $(-1,-1)$, $(-1,1)$,
\item $\mathbb{R}x\mathbb{R}$: $(a,b)$ st $a,b \ne 0$
\end{itemize}

Fact:\\
Multiplicative Inverse:work in commutative rings\\
$\left(\begin{array}{l l}
1 & 2\\
3 & 4\\
\end{array}\right)$
$\cdot$
$\left(\begin{array}{l l}
1 & 0\\
0 & 1\\
\end{array}\right)$
=
$\left(\begin{array}{l l}
1 & 2\\
3 & 4\\
\end{array}\right)$\\

Claim: Multiplicative Inverses: if they exist, they are unique.\\
\begin{proof}
Suppose that for every element of ring R: $a$, there exists an element of R: $b$ st $ab = u$, the unit of the ring. Suppose that there exists another element of R: $c$ such that $ac = u$. This implies that $ab = ac$. This implies that $b = c$. Therefore, b is unique.\\
\end{proof}
Note: The inverse of a (if it exists) is unique and denoted by $a^{-1}$

\begin{definition}
Denote set of units in ring R by $U(R)$\\
ex.\\
$U(\mathbb{Z}) = \{1, -1\}$\\
$U(\mathbb{Q}) = \mathbb{Q}\backslash \{0\}$\\
$U(\mathbb{Z}_6) = \{1, 5\}$
\end{definition}

The set $U(R)$ has nice properties:\\
First:\\
Check $U(R)$ is closed under multiplication.
\begin{proof}
Suppose $a, b \in U(R)$. This implies that since $U(R)$ is closed under multiplication, $ab\in U(R)$. This implies that $ab$ is a unit.
\end{proof}
\section{Fields}
\begin{definition}
Field: A commutative ring with unity in which every non-zero element is a unit.
\end{definition}
\begin{definition}
Field: A commutative ring in which one can always solve equations of the form $ax=b$ where $a\neq 0$. And the solution is $x = a^{-1}b$.
\end{definition}
\begin{definition}
In all rings, we can add, subtract, and multiply.\\
In fields, we can divide.
\end{definition}
Fact: Every field is a domain.
\begin{theorem}
A Field has no zero divisors.
\end{theorem}
\begin{proof}
Suppose F is a field and $a \in F$, $a \neq 0$. Now suppose $ab = 0$. This implies that $aba^{-1} = b = 0(a^{-1}) = 0$ Therefore there are no zero divisors in F.
\end{proof}

\subsection{Finite Fields}
\begin{theorem}
$\mathbb{Z}_n$ is a field iff n is prime.
\end{theorem}

\begin{proof}
Forward Direction:\\
Suppose n is composite. This implies that there exists $a,b \in \mathbb{Z}_n$ such that $ab = n$. This implies that $[a][b] = [n] = [0]$. This implies that a and b are zero divisors and therefore $\mathbb{Z}_n$ is not a field.\\
Backwards Direction:\\
Suppose n is prime and let $0 < x < n$. This implies that there exists an inverse $[x]^{-1}$. That is, we must find a $y$ such that $[x][y] = [1]$. Because n is prime, we know $gcd[n,x] = 1$. By GCD identity, there exists  r,s such that $1 = rn + sx$ But then $[s][x] = [1] - [r][n] = [1]$. And so, $[s]$ is the inverse of $[x]$
\end{proof}

Note: There is an implication of the above proof we can use it to compute multiplicative inverses in $\mathbb{Z}_p$ where p is prime.

\begin{enumerate}
\item{Compute $[23]^{-1}$ in $\mathbb{Z}_{119}$\\


	Apply Euclid's algorithm to obtain:\\
    $119 = 23(5) + 4$\\
    $23 = 4(5) + 3$\\
    $4 = 3(1) + 1$\\
    $3 = 1(3) + 0$\\
    $gcd(119,23) = 1$\\
	$1 = 23(-31) + 119(6)$\\

    $1 = 0 + 23(-31)$ in $\mathbb{Z}_{119}$
    so $-31 = 88$ is the inverse of $23$ in $\mathbb{Z}_{119}$

}
\end{enumerate}

\subsection{Recipe to compute Inverse in $\mathbb{Z}_{p}$}
$[x]^{-1}$ in $\mathbb{Z}_{p}$
\begin{enumerate}
\item Given $x$ and $p$ using Euclid's algorithm to show $gcd(p,x) = 1$
\item Work backward through resulting equation to obtain $r,s$ for linear combination.
\item Reduce $s$ in $\mathbb{Z}_{p}$
\end{enumerate}

Note: this method works as long as $gcd(p,x) = 1$ aka they are relatively prime.\\

\subsection{Fermat's Little Theorem}
\begin{theorem}
(Alternative Approach to computing $[x]^{-1}$ in $\mathbb{Z}_{p}$)\\
If p is prime and $0 < x < p$, then $x^{p-1} \equiv 1$ in $\mathbb{Z}_{p}$. Hence, in $\mathbb{Z}_{p}$, $[x]^{-1} \equiv  [x^{p-2}]$
\end{theorem}

Example: in $\mathbb{Z}_{5}$ this $th^{m}$ claims that $p[3^4] \equiv [1]$ which implies that $[3]^{-1} \equiv [3]^{3} \equiv [27] \equiv [2]$

\begin{proof}
Suppose p is prime and $0 < x < p$. Then $[x]$ is a non-zero element of the field $\mathbb{Z}_p$. Consider the set $S$ of non-zero multiples of $[x]$ in $\mathbb{Z}_p$. $S = \{ [x*1],[x*2],[x*3]...[x*(p-1)]\}$. Because the field has no zero divisors, each element of S is non zero. Because a field satisfies multiplicative cancellation, no two of these elements are the same. Thus, the set $S$ consists of $p-1$ distinct non-zero elements and so must consist of the set $\{ [1], [2],..[p-1]\}$. We now multiply all elements of $S$ together as multiplying all non-zero elements of $\mathbb{Z}_p$. $[1][2]...[p-1][x]^{p-1} = [1][2][3]...[p-1]$. By multiplicative cancellation in the domain $\mathbb{Z}_p$ we can cancel $[1][2]...[p-1]$ from each side, so $[x]^{p-1} \equiv 1$ (mod $p$).
\end{proof}

Check the key idea of FLT for x=3 and p=17 ie. check that the set S consists of 16 non-zero elements of $\mathbb{Z}_p$.\\
$\{[3*1], [3*2], [3*3], [3*4], [3*5], [3*6], [3*7], [3*8], [3*9], [3*10], [3*11], [3*12], [3*13], [3*14], [3*15], [3*16]\} \equiv \{ [3], [6], [9], [12], 15], [1], [4], [7], [10], [13], [16], [2], [5], [8], [11], [14]\}$


\section{Differential Equations}
\subsection{Function Definition}
Correspondence between domain $D$ and range $R$ such that $\exists d \in D $ for $\exists r \in R$ that is unique.\\

\subsubsection{Note:}

If $y = ln(x)$, successive derivatives are:
$y'=\frac{1}{x}$, $y'=\frac{-1}{x^2}$, $y'=\frac{2}{x^3}$

If $z= x^3 -3xy+2y^2$, its partial derivatives with respect to $x$ and with respect to $y$ are respectively: $\frac{\partial z}{\partial x} = 3x^2 - 3y$ and $\frac{\partial z}{\partial y} = -3x+4y$.

The $2^{nd}$ partial derivateves are respectively: $\frac{\partial^2 z}{\partial x^2} = 6x$ and $\frac{\partial^2 z}{\partial y^2} = 4$

Both sets of equations in this note are both differential equations, the first is an example of ordinary differential equations: ODEs, the second is an example of partial differential equations: PDE


\textbf{ODE Defn: equation involving $x$, a function: $f(x)$, and one or more of its derivatives}

\subsubsection{Examples:}

\begin{enumerate}
\item $\frac{dy}{dx} + y = 0$
\item $y' = e^x$
\item $\frac{d^2 y}{d x^2} = \frac{1}{1-x^2}$
\item $f'(x) = f''(x)$
\item $x y' = 2y$
\item $y'' + (3y')^3 +2x = 7$
\item $(y''')^2 + (y'')^4 + y' = x$
\item $xy'''' + 2y''+(xy')^5 = x^3$
\end{enumerate}

\subsubsection{Definition: Order of Differential Equation}
order of the highest derivative involved in the equation

First Order: $1$, $2$, $5$

Second Order: $3$, $4$, $6$

Third Order: $7$

Fourth Order: $8$

\vspace{6pt}

\subsubsection{Example:}

$y'' - y'' + y' - y = 0$

Solution of Differential Equations:\\

Consider: $x^2 - 2x - 3 = 0$, $x = 3$ is a solution\\

Now Consider: (1) $y=f(x)=ln(x)+x$, $x > 0$ and (2) $x^2y'' + 2xy' y = ln(x) 3x + 1$\\

(1) is a solution to (2)\\

\subsubsection{Note:}
\begin{enumerate}
\item We specified the values of x for which function is defined
\item Specified interval where Differential Equation makes sense
\end{enumerate}


\subsubsection{Definition: Explicit Solution}
Let $y=f(x)$ be a function of x on $I: a<x<b$, we say $f(x)$ is an EXPLICIT SOLUTION or a SOLUTION of an ODE evolving $x, f(x),$ and its derivatives if it satisfies the equation for \textbf{every} $x \in I$.














\end{document}











\e
